% Preambuła
\documentclass[a4paper,11pt]{article}
\usepackage[polish]{babel}
\usepackage[OT4]{fontenc}
\usepackage[utf8]{inputenc}
\usepackage{fancyhdr}
\usepackage{listings}
\usepackage{amsfonts}
\usepackage{bm}

\pagestyle{fancy}
\lhead{ZO'2013 - zadanie domowe nr 4}
\rhead{Wojciech Żółtak; 292583}

\lstset{numbers=left}

% Część główna
\begin{document}

Nierówną walkę z problemem, który więcej wydaje się mieć więcej wspólnego z
rachunkiem prawdopodobieństwa niż złożnością obliczeniową, zaczniemy od prostej
obserwacji. Treść zadania mówi, że algorytm musi działać poprawnie wyłącznie dla
grafów 3-kolorowalnych. Znaczy to, że możemy założyć o grafie wejściowym, że
3-kolorowalny jest. W najgorszym wypadku damy złą odpowiedź, ale przecież nam
wolno.

Jak się tak zastanowić, to z klasycznego 3-kolorowania do 2-kolorowania bez
trójkątów monochromatycznych droga jest dość prosta. Każdy trójkąt z definicji
3-kolorowania musi być pokolorowany na 3 różne kolory. Zatem wystarczy np.
utożsamić dwa dowolne kolory i już! Wyszło. Szkoda tylko, że problem znalezienia
3-kolorowania jest NP-trudny... \\


\subsection*{Szukając na ślepo}

Spróbujmy trochę inaczej. Dla ustalenia uwagi - powiedzmy, że mamy 3
kolory - R, G, B, z których dwa pierwsze wykorzystujemy do 2-kolorowania.
2-kolorowanie jest ,,dobre'', jeśli zachowuje kolory R i G z 3-kolorowania
na wierzchołkach, które leżą na jakimś trójkącie
(wierzchołki, które miały kolor B mogą być pokolorowane dowolnie, podobnie
wierzchołki poza trójkątami).


Załóżmy, że mamy jakieś 2-kolorowanie grafu, które nie jest dobre. Zauważmy przy
tym, że kolorowania mogą być w różny sposób ,,niedobre'', tzn. niektóre w jakiś
sposób mogą być lepsze od innych (mimo bycia złymi). Spróbujmy to jakoś
sformalizować - intuicyjnie 2-kolorowanie jest tym lepsze im mniej trzeba w nim
poprawić, żeby uzyskać kolorowanie, które jest rozwiązaniem problemu. Niech
zatem ,,dobroć'' rozwiązania będzie ilością wierzchołków leżących na jakimś
trójkącie, posiadających w 3-kolorowaniu kolory R i G, które w naszym 2-kolorowaniu
są pomalowane odwrotnie. Jest to ilość ,,kroków'' (pojedynczych zamian kolorów)
jakie trzeba wykonać, żeby dojść do rozwiązania bazującego na tym konkretnym
3-kolorowaniu. 

Niepokojące może się wydawać, że odnosimy się do obiektu, którego nie znamy
(,,dobrego'' kolorowania). Istotnie - nie znamy jego postaci, więc nie możemy
zmierzyć dokładnie ,,jak dobre'' jest nasze kolorowanie. Możemy się za to
zastanawiać jaką mamy szansę się do niego zbliżyć próbując coś poprawić. \\


Przyjrzyjmy się naszemy 2-kolorowaniu. Ponieważ jest złe, to posiada trójkąt
monochromatyczny. Dla ustalenia uwagi załóżmy, że ma on kolor R.
Może da się go poprawić? Spróbujmy zamienić w nim kolor losowo
wybranego wierzchołka i zastanówmy się jaki ma to wpływ na jakość rozwiązania.
W zależności od tego jaki kolor wierzchołek ma w 3-kolorowaniu, na którym bazuje
szukane przez nas rozwiązanie mogą stać się różne rzeczy:
\begin{itemize}
  \item{Dla B nic się nie zmienia w kwestii jakości, ponieważ z definicji
        wierzchołki takie nie mają na nią wpływu.}
  \item{Dla R oddalimy się od rozwiązania o 1, ponieważ zmienimy kolor
        wierzchołka na niewłaściwy}
  \item{Dla G przybliżymy się do rozwiązania o 1, ponieważ poprawimy kolor
        wierzchołka na pożądany}
\end{itemize}
Co prawda nie wiemy jak dokładnie ułożone są wierzchołki w 3-kolorowaniu, ale
nic nam to nie psuje - ponieważ losujemy, prawdopodobieństwo wylosowania danego
wariantu wynosi dokładnie $\frac{1}{3}$. Sytuacja dla trójkąta w kolorze G jest
zupełnie analogiczna - wierzchołki R i G zamieniają się rolami, ale pod względem
prawdopodobieństwa poprawienia/pogorszenia rozwiązania operacja zachowuje się
tak samo. \\


No dobrze, to podsumujmy - wybierając losowy trójkąt monochromatyczny w złym
2-kolorowaniu i zamieniając w nim kolor losowego wierzchołek możemy przybliżyć
się do rozwiązania o 1 krok, oddalić o 1 krok lub nic nie zmienić, z
prawdopodobieństwiem $\frac{1}{3}$ każde. Wiemy też, że odległość dowolnego
złego 2-kolorowania od rozwiązania jest nie większa niż ilość wierzchołków $N$.
Ale zaraz... coś tu nie gra. Przed chwila udowadnialiśmy, że jeśli rozwiązanie
jest ,,niedobre'' to możemy je pogorszyć, a teraz twierdzimy, że jakość
kolorowania ma ograniczenie dolne. Odpowiedź jest prosta - nasze kolorowanie
możemy zepsuć tak bardzo, że aż stanie się dobre! Jeśli wszystkie wierzchołki R
i G są źle pomalowane, to znaczy, że zamieniliśmy kolory ze sobą, czyli
uzyskaliśmy rozwiązanie dualne do naszego pierwotnego. Nie można bardziej
,,popsuć'' rozwiązania, bo przesłanka o jego niepoprawności nie jest spełniona.
Ponadto, jeśli podgraf złożony wyłącznie z trójkątów nie jest spójny, to może
się okazać, że nawet mając ,,jakość'' gdzię między $0$ a $n$ i tak trzymamy w
ręku dobre rozwiązanie. Nie jest to jednak dla nas problem - możemy z
powodzeniem założyć pesymistycznie, że na takowe nie wpadniemy i z pewnością nasze
dalsze szacunki nie stracą swojej mocy. \\



\subsection*{Spacer losowy}

Jeśli będziemy potwarzać operację ,,poprawiania'' wielokrotnie, będziemy
,,błądzili'' po odcinku od $0$ do $n$ opisującym odległość od rozwiązania
pewnego rozwiązania (chociaż nie wiemy jakiego). Jest to
zatem wariacja na temat spaceru losowego po odcinku, w którym z
prawdopodobieństwem $\frac{1}{3}$ idziemy w prawo, w lewo, bądź pozostajamy w
miejscu. Nasz odcinek ma bariery pochłaniające na końcach, które wiążą się ze
znalezieniem rozwiązania. To czego będziemy szukać, to wartość oczekiwana ilości
kroków potrzebnych do dotarcia do którejś z barier z losowego punktu na odcinku.
Aby to policzyć rozpiszemy sobie układ równań: \\

\begin{eqnarray}
  \nonumber E_0 & = & 0 \\
  \nonumber E_i & = & \frac{1}{3}(E_{i-1} + E_{i} + E_{i+1}) + 1
  \ \ \ \mbox{(dla $0 < i < n$)}\\
  \nonumber E_n & = & 0
\end{eqnarray}

Rozpisując środkowe równanie otrzymujemy:

\begin{eqnarray}
  \nonumber \frac{2}{3} E_i & = & \frac{1}{3}(E_{i-1} + E_{i+1}) + 1 \\
  \nonumber E_i & = & \frac{1}{2}(E_{i-1} + E_{i+1}) + \frac{3}{2}
\end{eqnarray}

Spróbujmy rozpisać parę kolejnych równań i może coś z tego wyniknie.

\begin{eqnarray}
  \nonumber h_1 & = & \frac{1}{2}(h_0 + h_2) + \frac{3}{2} \\
  \nonumber h_1 & = & \frac{h_2}{2} + \frac{3}{2} \\
  \nonumber & & \ \\
  \nonumber & & \ \\
  \nonumber h_2 & = & \frac{1}{2}(h_1 + h_3) + \frac{3}{2} \\
  \nonumber h_2 & = & \frac{1}{2}(\frac{h_2}{2} + h_3 + \frac{3}{2}) + \frac{3}{2} \\
  \nonumber \frac{3}{4}h_2 & = & \frac{h_3}{2} + \frac{9}{4} \\
  \nonumber h_2 & = & \frac{2}{3}h_3 + 3
\end{eqnarray}

Naprawdę nie chcemy tego dalej rozpisywać. Spróbujmy zgadnąć jak to się rozwija
dalej i udowodnić indukcyjnie. Udowodnimy indukcyjnie, że dla równań opisujących
środek ścieżki zachodzi:

\begin{eqnarray}
  \nonumber h_{i} = \frac{i}{i+1}h_{i+1} + \frac{3i}{2}
\end{eqnarray}

Pierwszy krok dowodu mamy zrobiony powyżej. Pozostaje udowodnić przejście między
kolejnymi $i$ korzystając z tezy indukcyjnej:

\begin{eqnarray}
  \nonumber h_{i + 1} & = & \frac{1}{2}(h_{i} + h_{i+2}) + \frac{3}{2}\\
  \nonumber h_{i + 1} & = & \frac{1}{2}(\frac{i}{i+1}h_{i+1} + \frac{3i}{2}
    + h_{i+2}) + \frac{3}{2} \\
  \nonumber h_{i + 1} & = & \frac{i}{2(i+1)}h_{i+1} + \frac{3i}{4}
    + \frac{h_{i+2}}{2} + \frac{3}{2} \\
  \nonumber \frac{i + 2}{2(i + 1)}h_{i+1} & = & \frac{h_{i+2}}{2}
    + \frac{3(i + 2)}{2} \\
  \nonumber h_{i+1} & = & \frac{i + 1}{i + 2}h_{i + 2} + \frac{3(i + 1)}{2}
\end{eqnarray}

Udało się! Ale to połowa roboty. Teraz należy wykonać drugi przebieg, tym razem
od końca, i wyliczyć konkretne wartości oczekiwane. Znów posłużymy się indukcją.
Nasza teza, dla $0 < i < n$ brzmi:

\begin{eqnarray}
  \nonumber h_{n-i} & = & \frac{3(n - i)i}{2}
\end{eqnarray}

Jest spełniona dla $i = 1$:

\begin{eqnarray}
  \nonumber h_{n-1} & = & \frac{n-1}{n}h_{n} + \frac{3(n-1)}{2} =
    \frac{3(n-1)}{2} = \frac{3(n-1)1}{2}
\end{eqnarray}

Więc spróbujmy jej użyć do wykonania kroku:

\begin{eqnarray}
  \nonumber h_{n-i-1} & = & \frac{n-i-1}{n-i}h_{n-i} + \frac{3(n - i - 1)}{2} \\
  \nonumber h_{n-i-1} & = & \frac{(n-i-1)3(n-i)i}{2(n-i)} +
    \frac{3(n - i - 1)}{2} \\
  \nonumber h_{n-i-1} & = & \frac{3(n-i-1)i}{2} + \frac{3(n - i - 1)}{2} \\
  \nonumber h_{n-i-1} & = & \frac{3(n-(i+1))(i + 1)}{2}
\end{eqnarray}

I wyszło. Mamy jawne wzory na wartości oczekiwane dla różnych miejsc startowych.
Nie wiemy jednak skąd będziemy startować, ani z jakim prawdopodobieństwem.
Dlatego zamiast wyliczać dokładną wartość oczekiwaną posłużymy się jej
pesymistycznym oszacowaniem. Nie bawiąc się w szczegóły zauważamy, że dla $0 < i
< n$ wynik jest zawsze ograniczony z góry przez $2n^2$.

Teraz, korzystając z nierówności Markowa:

\begin{eqnarray}
  \nonumber P(|X| \ge \epsilon) & \le & \frac{E(|X|)}{\epsilon} \\
  \nonumber P(|X| \ge 8n^2) & \le & \frac{2n^2}{8n^2} = \frac{1}{4}
\end{eqnarray}

Powyższy napis znaczy tyle, że prawdopodobieństwo, iż po $8n^2$ krokach wciąż
nie znajdziemy rozwiązania jest mniejsze, bądź równe $\frac{1}{4}$. Zatem
wykonanie $8n^2$ kroków gwarantuje dostatecznie dużą skuteczność algorytmu.


\subsection*{Podsumowując}

Nasz algorytm działa w następujący sposób:

\begin{itemize}
  \item{Losuje 2-kolorowanie grafu}
  \item{Dopóki nie uda się uzyskać poprawnego 2-kolorowania (ale nie dłużej niż
        $8n^2$ iteracji):
    \begin{itemize}
      \item{Wybiera losowy trójkąt monochromatyczny}
      \item{Zmienia w nim kolor losowego wierzchołka}
    \end{itemize}
  }
  \item{Jeśli udało się znaleźć 2-kolorowanie, to zwraca je jako wynik, jeśli
        nie to stwierdza, że ono nie istnieje.}
\end{itemize}

Wylosowanie 2-kolorowania, sprawdzenie jego poprawności, znalezienie trójkątów
monochromatycznych i zamiana koloru wierzchołka są operacjami wielomianowymi.
Maksymalna ilość iteracji jaką wykonuje algorytm też jest wielomianowa, zatem
cały działa w czasie wielomianowym.

Jeśli graf jest 3-kolorowalny, to z prawdopodobieństwem $\frac{1}{4}$ nie uda nam
się znaleźć rozwiązania.

Zatem, wszystko jest tak jak być powinno.

\end{document}
