% Preambuła
\documentclass[a4paper,11pt]{article}
\usepackage[polish]{babel}
\usepackage[OT4]{fontenc}
\usepackage[utf8]{inputenc}
\usepackage{fancyhdr}
\usepackage{listings}
\usepackage{amsfonts}
\usepackage{bm}

\pagestyle{fancy}
\lhead{ZO'2013 - zadanie domowe nr 3}
\rhead{Wojciech Żółtak; 292583}

\lstset{numbers=left}

% Część główna
\begin{document}

\section*{Zacznijmy od czegoś prostego, czyli $k=0$}

Żeby wyrobić sobie intuicję co do problemu, naszych własnych potrzeb oraz
możliwości, spróbujmy rozwiązać problem w wersji ztrywializowanej - dla maszyny
alternującej, która nie wykonuje alternacji. Istnieją dwa rodzaje takich maszyn:
\begin{itemize}
  \item{w których osiągalne są wyłącznie stany egzystencjalne}
  \item{w których osiągalne są wyłącznie stany uniwersalne}
\end{itemize}

W pierwszym przypadku maszyna alternująca odpowiada niedeterministycznej
maszynie turinga. Wiadomo również, że $PSPACE = NPSPACE$, zatem trywialnie język
rozpoznawany przez maszynę należy do $PSPACE$. \\


Rozpatrzmy maszynę, w której osiągalne są wyłącznie stany uniwersalne. Na
pierwszy rzut oka sytuacja wydaje się dość nieciekawa - żeby ,,zasymulować''
maszynę alternującą pracującą w stanach uniwersalnych na niedeterministycznej
maszynie turinga wypadałoby pamiętać bardzo dużo rzeczy - np. zbiór stanów, w
których się znajdujemy razem z odpowiadającymi im taśmami. To stanowczo
zbyt wiele dla pamięci wielomianowej.

Intuicyjnie, aby udowodnić, że słowo należy do języka należy sprawdzić
,,wszystko''. Dużo łatwiej pokazać, iż słowo do niego nie należy - wystarczy
znaleźć ,,coś'' - obliczenie, które jest kontrprzykładem. Z definicji maszyny
alternującej  warunkiem koniecznym i dostatecznym do odrzucenia słowa jest
istnienie
\begin{itemize}
  \item{obliczenia prowadzącego do bezpośredniego odrzucenia wejścia lub}
  \item{nieskończonego (pętlącego się) obliczenia.}
\end{itemize}

To co chciałoby się zrobić, to skonstruować niedeterministyczną maszynę turinga,
która akceptuje wejście wówczas, gdy uda się jej ,,zgadnąć'' odrzucające
obliczenie, a następnie zanegować jej odpowiedź (tzn. akceptować wejście gdy nie
znajdzie się kontrprzykładu). Problem sprawiają jednak obliczenia nieskończone -
maszyna powinna zaakceptować gdy zapętli się, ale.. no właśnie, pętli się, więc
nie akceptuje. Spróbujmy wyeliminować ten problem konstruując dla naszej maszyny
alternującej maszynę totalną.


\section*{Alternująca maszyna totalna dla $L(M_a) \in PSPACE$}

Kiedy wiadomo, że obliczenie zapętla się? Gdy powtórzy się konfiguracja maszyny.
Kiedy wiadomo, że konfiguracja się powtórzyła? Nasza maszyna pracuje w pamięci
wielomianowej, którą można oszacować z góry przez jakiś wielomian $w(n)$. Niech
$A$ będzie rozmiarem alfabetu (wliczając blanki), a $Q$ łączną ilością stanów
egzystencjalnych i uniwersalnych maszyny. Ilość konfiguracji jakie może
wygenerować maszyna dla wejścia długości $n$ jest zatem nie większa niż
$A^{w(n)} * w(n) * Q$ - taśma zapisana jest jakimiś symbolami, gdzieś znajduje
się głowica i jesteśmy w jakimś stanie. To dość spora liczba, ale zapisana
binarnie z powodzeniem zmieści się w pamięci wielomianowej.

Mając maszynę alternującą $M_a$ pracującą w pamięci wielomianowej możemy ją
przerobić w taki sposób, żeby
\begin{itemize}
  \item{zliczała kroki wykonywane w ,,pierwotnym'' obliczeniu (przed przeróbką),}
  \item{nie zapisywała na taśmę blanków (można dodać nowy symbol do alfabetu,
        który w pierwotnym obliczniu maszyny jest interpretowany jak blank,
        chociaż formalnie nim nie jest i jest zapisywany na taśmę w momencie,
        gdy pierwotna maszyna zapisałaby blank),}
  \item{zliczała ilość pamięci zużywanej na pierwotne obliczenie (poprzez
        zliczanie napotkanych blanków - dlatego chcieliśmy uniemożliwić
        zapisywanie ich na taśmę przez maszynę)}
  \item{dla każdego nowego oszacowania zużywanej pamięci wylicza nowy limit
        kroków}
  \item{jeśli ilość kroków przekroczy aktualny limit obliczenie jest przerywane a wejście
        odrzucane}
\end{itemize}
Ponieważ maszyna w momencie gdy się zapętla nie zwiększa ilości zużywanej
pamięci, to nie jest ona w stanie zwiększyć aktualnego limitu, jednocześnie
ciągle wykonując kolejne kroki. Zatem z pewnością się zatrzyma. Natomiast
skończone obliczenie nie zostanie przedwcześnie ,,wywłaszczone'' ponieważ
niezmiennie limit kroków dostosowany jest do aktualnie wykorzystywanej pamięci i
jego przekroczenie oznaczałoby zapętlenie, a więc sprzeczność. \\ \\

Zatem, maszyna taka jest totalna.


\section*{Wracając do początku}

Mając do dyspozycji konwersję do maszyny totalnej możemy dokończyć nasze
wcześniejsze rozwiązanie dla $k=0$. Dla maszyny, w której w obliczeniu osiągalne
są wyłacznie stany uniwersalne bierzemy jej wersję totalną, a następnie za
pomocą niedeterministycznej maszyny turinga zgadujemy obliczenie, które kończy
się odrzuceniem. W przypadku gdy się to uda - odrzucamy słowo wejściowe, w
przypadku przeciwnym, akceptujemy. Ponieważ $PSPACE = NPSPACE$ wszystko jest w
porządku.

Ponadto, bez straty ogólności, możemy dodatkowo przyjąć, że nasza maszyna
startuje w dowolnej konfiguracji, byle tylko nie wykonywała żadnej alternacji
w trakcie obliczeń. \\


Przypadek $k=0$ wydaje się być dobrą podstawą do indukcji po $k$. Załóżmy zatem,
że umiemy rozwiązać nasz problem dla $k-1$ i spróbujmy za jego pomocą zbudować
rozwiązanie dla $k$.

Ponownie, istnieją dwa przypadki:
\begin{itemize}
  \item{maszyna startuje w stanie egzystencjalnym lub}
  \item{maszyna startuje w stanie uniwersalnym}
\end{itemize}

Nasze rozwiązanie dla $k-1$ przemieńmy w wyrocznię, która dla zadanej
konfiguracji maszyny, przy założeniu, że nie wykona ona więcej niż $k-1$
alternacji, umie stwierdzić czy dalsze obliczenie zakończy się akceptacją czy
odrzuceniem.

W przypadku gdy startujemy w stanie egzystencjalnym niedeterministyczna maszyna
turinga może zgadywać ścieżkę do stanu akceptującego. Jeśli natrafi na stan, w
którym następuje alternacja wystarczy, że zapyta wyrocznię czy dalsze obliczenie
kończy się sukcesem. Wówczas akceptujemy.

W przypadku gdy startujemy w stanie uniwersalnym, postępujemy podobnie. Nasza
maszyna będzie wykorzystywać maszynę, która zgaduje obliczenie-kontrprzykład. W
momencie alternacji pyta wyrocznię, czy dalsze obliczenie na maszynie
alternującej odrzuciłoby wejście. Jeśli tak, to maszyna pomocnicza akceptuje.
Nasza maszyna akceptuje wówczas gdy maszyna pomocnicza tego NIE robi.

W obu przypadkach korzystamy z wyroczni - działającej w $PSPACE$ oraz symulujemy
obliczenie w maszynie alternującej, które z założenia jest w $PSPACE$. Zatem
całość też jest w $PSPACE$.\\


Ponownie, problemem może być pętlenie się maszyny w obrębie pojedynczej
,,warstwy''. I ponownie - możemy użyć wersji totalnej naszej maszyny
alternującej.

Dodatkowo, bez straty ogólności możemy założyć, że nasza maszyna startuje z
dowolnej konfiguracji, o ile dowolne obliczenie nie spowoduje wykonania wiecej
niż $k$ alternacji.



\section*{Podsumowując - indukcja}

Podsumowując powyższe rozważania:
\begin{itemize}
  \item{Umiemy rozwiązać nasz problem w $PSPACE$ dla $k=0$.}
  \item{Zakładając, że umiemy rozwiązać nasz problem w $PSPACE$ dla $k-1$
        umiemy skonstruować rozwiązanie dla $k$, które jest w $PSPACE$.}
\end{itemize}

Z indukcji, dla dowolnego naturalnego $k$, dla języka $L$ rozpoznawanego w
pamięci wielomianowej przez maszynę alternującą maksymalnie $k$ razy
zachodzi $L \in PSPACE$.



\end{document}
