% Preambuła
\documentclass[a4paper,11pt]{article}
\usepackage[polish]{babel}
\usepackage[OT4]{fontenc}
\usepackage[utf8]{inputenc}
\usepackage{fancyhdr}
\usepackage{listings}
\usepackage{amsfonts}
\usepackage{bm}

\pagestyle{fancy}
\lhead{ZO'2013 - zadanie domowe nr 2}
\rhead{Wojciech Żółtak; 292583}

\lstset{numbers=left}

% Część główna
\begin{document}

\section*{Opis problemu}

Aby udowodnić $P$-zupełność problemu sprawdzenia niepustości języka generowanego
przez gramatykę bezkontekstową muszę:

\begin{itemize}
  \item{sformalizować kodowanie wejścia (gramatyki)}
  \item{pokazać, że problem w ogóle należy do klasy $P$}
  \item{udowodnić zupełność w klasie $P$}
\end{itemize}


\section*{Kodowanie gramatyk bezkontekstowych}

Kodowanie gramatyki przedstawia się następująco:
\begin{itemize}
  \item{Symbolom nieterminalnym przyporządkowujemy ciągi zer nieparzystej
        długości, przy czym symbolowi startowemu odpowiada $0$.}
  \item{Symbolom terminalnym przyporządkowujemy ciągi zer parzystej długości.}
  \item{Produkcja ma postać kodów symboli rozdzielonych jedynkami, tzn.
        produkcja postaci $A \rightarrow x_{1}..x_{k}$ kodowana jest jako
        $K_{A} 1 K_{x_{1}} 1 .. 1 K_{x_{k}} $, gdzie $K_x$ jest kodem symbolu
        $x$.}
  \item{Produkcje rozdzielone są parami jedynek.}
  \item{Wszystkie produkcje okalają dwie trójki jedynek.}
\end{itemize}
Jest to zatem kalka z kodowania maszyn turinga z wykładu, z drobną optymalizacją
pozwalającą na rozpoznawanie czy czytany symbol jest terminalny czy nie (po
parzystości).

Wszystkie operacje (przeczytanie produkcji dla danego nieterminala, sprawdzenie
czy symbol jest termilanem czy nie, znalezienie wszystkich symboli) można
w naiwny sposób wykonać w czasie wielomianowym od rozmiaru wejścia.



\section*{Przynależność do klasy $P$}

Algorytm sprawdzający niepustość języka generowanego przez daną gramatykę
wygląda następująco:

\begin{itemize}
  \item{Tworzymy pusty zbiór $Productible$}
  \item{Tak długo, aż zbiór $Productible$ przestanie się powiększać:
    \begin{itemize}
      \item{Niech $Prod'$ będzie pustym zbiorem}
      \item{Dla każdego nieterminala $A$ z gramatyki:
        \begin{itemize}
          \item{Jeśli istnieje produkcja postaci $A \rightarrow x_{1} .. x_{k}$
                taka, że
\begin{displaymath}
  \forall_{i \in {1..k}} terminal(x_{i}) \vee (\neg terminal(x_{i}) \wedge x_{i} \in Productible)
\end{displaymath}

            to dodajemy $A$ do $Prod'$}
        \end{itemize}
      }
      \item{Dokładamy zawartość $Prod'$ do $Productible$}
    \end{itemize}
  }
  \item{Język jest niepusty, jeśli symbol startowy należy do $Productible$}
\end{itemize}


Zbiór $Productible$ można interpretować jako zbiór nieterminali, z których da
się wyprowadzić jakieś słowo. Wypadałoby pokazać jego poprawność i pełność.

Jako $P_{n}$ oznaczmy zbiór nieterminali z gramatyki, z których można
wyprodukować słowo za pomocą drzewa wyprowadzenia o wysokości nie większej niż
$n$.

$P_{0}$ jest puste.

$P_{1}$ zawiera wszystkie nieterminale posiadające produkcję, która z
prawej strony posiada wyłącznie terminale. Po pierwszej iteracji algorytmu
zachodzi $P_{1} = Productible$, z definicji algorytmu.

Rozważymy zbiór $P_{n+1}$ i wybierzmy z niego dowolny nieterminal $A$. Z
definicji tego zbioru istnieje słowo $w$, która da się wyprowadzić z $A$ za
pomocą drzewa wyprowadzenia o wysokości nie większej niż $n+1$. W korzeniu tego
drzewa użyta jest pewna produkcja z $A$. Dla każdego symbolu $x$ z prawej strony
tej produkcji
\begin{itemize}
  \item{$x$ jest terminalem lub}
  \item{$x$ jest nieterminalem i $x \in P_{n}$.}
\end{itemize}
Druga własność wynika z tego, że gdyby $\neg terminal(x) \wedge x \notin P_{n}$,
to całe drzewo wyprowadzenia musiało by mieć wysokość większą niż $n+1$. Co jest
sprzeczne z założeniem o $P_{n+1}$.

Powyższa obserwacja w jasny sposób przekłada się na działanie algorytmu.
Zakładając, że na początku pewnej iteracji $P_{n} \subseteq Productible$, to po
jej zakończeniu $P_{n+1} \subseteq Productible$, ponieważ dla dowolnego nieterminalu
$A \in P_{n+1}$ produkcja z korzenia drzewa wyprowadzenia spełni warunek z
algorytmu i $A$ zostanie dodane do $Productible$.

Co więcej, w $k$-tym kroku zbiór $Productible$ wzbogaca się tylko o symbole
należące do $P_{k}$, ponieważ jeśli dodajemy do $Productible$ symbol $A$,
to znaczy, że istnieje z niego produkcja w symbole będące terminalami, bądź
należące do $P_{k-1}$, zatem z $A$ można wyprowadzić słowo drzewem wysokości nie
większej niż $k$ za pomocą tej produkcji.

Zatem, z indukcji - po $k$-tej iteracji algorytmu zachodzi
$P_{k} = Productibles$. \\


No dobrze, ale ile jest tych iteracji? W szczególności - czy algorytm w ogóle
się zatrzyma? Zuważmy, że:
\begin{itemize}
  \item{Jeśli dla pewnego $k$ zachodzi $P_{k} = P_{k+1}$, to $\forall_{i > k}
        P_{k} = P_{i}$, ponieważ przy ustalonej gramatyce wynik kolejnej
        iteracji algorytmu zależy wyłącznie od wyniku poprzedniej.}
  \item{Moc zbiorów $P_{k}$ jest ograniczona z góry przez moc zbioru symboli
        nieterminalnych gramatyki - $M$.}
\end{itemize}
W konsekwencji, po najwyżej $M$ iteracjach zbiór $Productible$ przestanie się
powiększać. Stanie się tak, ponieważ
\begin{itemize}
  \item{jeśli wystąpi iteracja, po której zbiór $Productible$ nie zwiększy się,
        to żadna kolejna iteracja nie doda do niego nowego elementu}
  \item{jeśli każda iteracja doda coś do $Productibles$, to po $M$ krokach
        będzie on zawierał wszystkie symbole nieterminalne, a więc nie będzie
        dało się go dalej powiększyć}
\end{itemize}

Zatem, po $M$ iteracjach algorytm zakończy się oraz zachodzić będzie

\begin{displaymath}
  \forall_{i \in \mathbb{N}} x \in P_{i} \Rightarrow x \in Productible
\end{displaymath}
  oraz
\begin{displaymath}
  \forall_{x \in Productible} \exists_{i \in \mathbb{N}} (x \in P_{i})
\end{displaymath}

czyli zbiór $Productible$ zawierać będzie tylko i wyłącznie symbole
nieterminalne, z których da się wyprowadzić jakieś słowo skończonym drzewem
wyprowadzenia.

W szczególności, język generowany przez gramatykę jest niepusty wtedy i tylko
wtedy gdy jej symbol startowy należy do $Productible$.\\


Niech $N$ będzie rozmiarem wejście.
Cały algorytm składa się z $\mathcal{O}(M) \leq \mathcal{O}(N) $ iteracji,
w których przeglądamy wszystkie $\mathcal{O}(N)$ produkcji, a w nich analizujemy
$\mathcal{O}(N)$ symboli, przeglądając przy tym zbiór $Productible$, którego
rozmiar również możemy przeszacować przez $\mathcal{O}(N)$.
Pojedyncze operacje na gramatyce i zbiorze $Productible$ są wielomianowe, więc
możemy je przeszacować przez $\mathcal{O}(P(N))$, gdzie $P(N)$ jest pewnym
wielomianem. Wobec tego, cały algorytm powinien działać w czasie 
$\mathcal{O}( N^5 * P(N))$, czyli w czasie wielomianowym.

Zatem nasz problem leży w klasie $P$.

\section*{Zupełność}

W celu udowodnieniu zupełności w klasie $P$ posłużę się redukcją z {\it Circuit
Value Problem} który jest $P$-zupełny. Dla przypomnienia - mając dany obwód oraz
wartościowanie wejść należy określić wartość logiczną w wyróżnionej bramce
wyjściowej.

Obwody z wykładu nie posiadały bramek NOT - zamiast tego każde wejście było
podawane w wersji normalnej i zanegowanej. Będę używał wersji {\it CVP} dla tego
rodzaju obwodów. Na koniec wspomnę jak by wyglądała by redukcja dla obwodów z
bramkami NOT. \\


Redukcja wygląda następująco:\\

\begin{itemize}
  \item{Niech $n$ oznacza ilość bramek w obwodzie, a $2m$ ilość wejść (łącznie,
        normalnych i zanegowanych).}
  \item{Niech $P_{1}$ do $P_{n}$ będą bramkami z obwodu, przy czym niech $P_{1}$
        będzie bramką wyjściową. Natomiast $x_1$ do $x_{2m}$ wartościami wejść.}
  \item{Dla każdej bramki $P_i$ generujemy produkcje:
    \begin{itemize}
      \item{$S_i \rightarrow S_k S_j$, jeśli $P_i$ jest postaci $P_k \wedge P_j$}
      \item{$S_i \rightarrow S_k V_j$, jeśli $P_i$ jest postaci $P_k \wedge x_j$
            lub $x_j \wedge P_k$}
      \item{$S_i \rightarrow V_k V_j$, jeśli $P_i$ jest postaci $x_k \wedge x_j$}

      \item{$S_i \rightarrow S_k$, $S_i \rightarrow S_j$, jeśli $P_i$ jest postaci $P_k \vee P_j$}
      \item{$S_i \rightarrow S_k$, $S_i \rightarrow V_j$, jeśli $P_i$ jest postaci $P_k \vee x_j$
            lub $x_j \vee P_k$}
      \item{$S_i \rightarrow V_k$, $S_i \rightarrow V_j$, jeśli $P_i$ jest postaci $x_k \vee x_j$}
    \end{itemize}
  }
  \item{Dla każdego wejścia $x_i$ (przypominam, że to zbiór zarówno wejść
        normalnych jak i zanegowanych) generujemy produkcje:
    \begin{itemize}
      \item{$V_i \rightarrow \epsilon $, gdy $x_i$ ma wartość $1$}
      \item{$V_i \rightarrow V_i$ w przeciwnym przypadku}
    \end{itemize}
  }
  \item{Produkcje wypisujemy na wyjście w takiej kolejności by te dla $P_{1}$
        znalazły się na początku ($S_{1}$ ma być symbolem startowym).}
\end{itemize}


Na początek zauważmy, że powyższa gramatyka generuje słowo puste lub nic
(ponieważ może zawierać cykle potencjalnie można budować nieskończenie długie
drzewa wyprowadzeń i do niczego nie dojść).

Dodatkowo, dla każdego nieterminala z powyższej gramatyki spełniony jest
niezmiennik:

{\it Z nieterminala można wyprowadzić słowo $\Longleftrightarrow$
 odpowiadający mu element obwodu (bramka lub wejście) zwraca wartość $1$.}

Można to pokazać indukcją strukturalną po obwodzie i odpowiednich
nieterminalach. W szczególności:
\begin{itemize}
  \item{Dla nieterminali $V_i$, jeśli odpowiadające wejście ma wartość:
    \begin{itemize}
      \item{$1$, to jedyna produkcja powoduje wyprowadzenie słowa pustego}
      \item{$0$, to nie da się nic wyprowadzić, bo jedyna produkcja pętli się}
    \end{itemize}
  }
  \item{Dla nieterminali $S_i$, jeśli odpowiada im bramka OR mająca wartość:
    \begin{itemize}
      \item{$1$, to jedno z wejść spięte jest z obwodem zwracającym $1$;
            odpowiadający mu nieterminal, według niezmiennika, pozwala
            wyprodukować słowo puste i możemy do niego przejść wybierając
            odpowiednią produkcję z $S_i$}
      \item{$0$, to do obu wejść wpięte są obwody zwracające $0$, w związku z
            czym, według niezmiennika, niezależnie od tego którą produkcję
            wybierzemy, nie zdołamy niczego wyprodukować}
    \end{itemize}
  }
  \item{Dla nieterminali $S_i$, jeśli odpowiada im bramka AND mająca wartość:
    \begin{itemize}
      \item{$1$, oba wejścia obwodu połączone są z obwodami zwracającymi $1$,
            w związku z czym z obu odpowiadających nieterminali można
            wyprowadzić słowo puste, w konsekwencji otrzymując słowo puste}
      \item{$0$, któreś z wejść obwodu połaczone jest z obwodem zwracającym $0$,
            w związku z czym z odpowiadającego mu nieterminala nie można
            wyprowadzić żadnego słowa, a w konsekwencji także z $S_i$}
    \end{itemize}
  }
\end{itemize}

Znaczy to ni mniej ni więcej tyle, że nasza język generowany przez powyższą
gramatykę jest niepusty (przy czym zawiera wtedy wyłącznie $\epsilon$) wtedy i
tylko wtedy gdy obwód zwraca wartość $1$. \\

Pozostaje uzasadnić, że redukcja jest wykonalna w pamięci logarytmicznej. Znaczy
to mniej więcej tyle, że mamy do dyspozycji skończoną liczbę liczników.
Wystarczy zauważyć, że o ile potrzebujemy ,,ponumerować'' bramki i wejścia, to
wcale nie musimy takiego mapowania pamiętać. Możemy wyliczać je dla
pojedynczych elementów ,,na żądanie'' przy pomocy stałej liczby liczników,
przeglądając całe wejście w wielomianowym czasie. Podczas generowania produkcji
dla dowolnego elementu obwodu musimy pamiętać wyłącznie numery oraz typy
obrabianego elementu oraz elementów wpiętych w wejścia, zatem stałą ilość
informacji. Tak więc całość można wykonać w pamięci logarytmicznej. \\

W związku z tym, nasz problem jest $P$-zupełny.



\subsubsection*{Wersja z NOT}

I na koniec szkic wersji dla obwodów zawierających bramki NOT. Rozwiązanie
wygląda podobnie jak w przypadku przekształcania obwodów z bramkami NOT na
postać z wykładów. Używa się w tym celu praw De Morgana w celu ,,zepchnięcia''
negacji do poziomu wyjść, a następnie dodaje zanegowane kopie wejść, których
można użyć na najniższym poziomie (samymi bramkami OR i AND nie da się generować
negacji).

Na poziomie gramatyki wygląda to w taki sposób, że każdy nieterminal
odpowiadający bramce ma swoją dualną wersję, która zwraca wartość zanegowaną.
Produkcje w takim dualnym nieterminalu są budowane według zasad dla przeciwnej
bramki (czyli w negacji ORa jest jedna produkcja w dwa nieterminale,
a w negacji AND'a są dwie produkcje w pojedyncze symbole), produkując przy tym
nieterminale dualne do symboli odpowiadającym obwodom wpiętym w wejścia.
Słowem, z praw De Morgana - zanegowanej bramce odpowiada bramka przeciwnego
typu z zanegowanymi wejściami.

Dla wejść natomiast generujemy dualne symbole odpowiadające ich zanegowanym
wartościom, co odpowiada podwojejeniu wejść przy konwersji obwodów.


\end{document}
