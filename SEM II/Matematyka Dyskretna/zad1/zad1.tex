% Preambuła
\documentclass[a4paper,11pt]{article}
\usepackage[polish]{babel}
\usepackage[OT4]{fontenc}
\usepackage[utf8]{inputenc}
\usepackage{fancyhdr}

\pagestyle{fancy}
\lhead{MD'2010 - zadanie domowe nr 1}
\rhead{Wojciech Żółtak}

% Część główna
\begin{document}
{\Large Oblicz:
\[
  \sum_{k=0}^{\infty} \frac{1}{(k+1)(k+3)(k+5)}.
\]}
\line(1,0){350}
\\
\\
\noindent Stosując metodę pobożnych życzeń będę chciał rozwinąć szereg 
w różnicę wyrazów, by móc go ''teleskopowo'' zwinąć do postaci zwartej. 
Mianownik sugeruje dwie oddzielne sumy, więc zaryzykuję policzenie następnego wyrażenia:
{\small \[
  (\frac{1}{k+1} - \frac{1}{k+3}) - (\frac{1}{k+3} - \frac{1}{k+5}) =
  \frac{1}{k+1} + \frac{1}{k+5}  - \frac{2}{k+3} =
  \frac{2(k+3)}{(k+1)(k+5)} - \frac{2}{k+3} =
\]
\[
  = \frac{2(k+3)^{2} - 2(k+1)(k+5)}{(k+1)(k+3)(k+5)} =
  \frac{2( (k^{2} + 6k + 9) - (k^2 + 6k + 5))}{(k+1)(k+3)(k+5)} = 
\]
\[
  = \frac{8}{(k+1)(k+3)(k+5)}
\]}
I w ten sposób, dzięki sporej ilości szczęścia dowiedziałem się, że:
{\small
\[
  \sum_{k=0}^{\infty} \frac{1}{(k+1)(k+3)(k+5)} =  \frac{1}{8} \sum_{k=0}^{\infty} \left( (\frac{1}{k+1} - \frac{1}{k+3}) - (\frac{1}{k+3} - \frac{1}{k+5}) \right)=
\]
{\tiny (zakładając, że szukana suma jest skończona)}
\begin{equation}
  = \frac{1}{8}\left(\sum_{k=0}^{\infty} (\frac{1}{k+1} - \frac{1}{k+3}) - \sum_{k=0}^{\infty} (\frac{1}{k+3} - \frac{1}{k+5})\right)\label{eq:rozklad}
\end{equation}
\\
Teraz z rozpisania kilku pierwszych wyrazów, skracania, wyciągania wniosków i przejścia granicznego:
{\small
\[
  \sum_{k=0}^{\infty} (\frac{1}{k+1} - \frac{1}{k+3}) = \frac{1}{1} - \frac{1}{3} + \frac{1}{2} - \frac{1}{4} + \frac{1}{3} - \frac{1}{5} + ... + \frac{1}{k+1} - \frac{1}{k+3} =
\]
\[
  = \lim_{k \to \infty}\left( \frac{1}{1} + \frac{1}{2} - \frac{1}{k+2} - \frac{1}{k+3}\right) = \frac{3}{2}
\]
{\tiny (oraz analogicznie)}
\[
  \sum_{k=0}^{\infty} (\frac{1}{k+3} - \frac{1}{k+5}) = \lim_{k \to \infty}\left( \frac{1}{3} + \frac{1}{4} - \frac{1}{k+4} - \frac{1}{k+5}\right) = \frac{7}{12}
\]
}

Zatem, ponieważ powyższe szeregi są zbieżne, (\ref{eq:rozklad}) faktycznie zachodzi i prawdziwa jest równość:

{\small
\[
  \sum_{k=0}^{\infty} \frac{1}{(k+1)(k+3)(k+5)} = \frac{1}{8} ( \frac{3}{2} - \frac{7}{12}) = \frac{11}{96}
\]
}

\end{document}

