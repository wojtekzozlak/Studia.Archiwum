% Preambuła
\documentclass[a4paper,11pt]{article}
\usepackage[polish]{babel}
\usepackage[OT4]{fontenc}
\usepackage[utf8]{inputenc}
\usepackage{fancyhdr}

\pagestyle{fancy}
\lhead{MD'2010 - zadanie domowe nr 5}
\rhead{Wojciech Żółtak}

% Część główna
\begin{document}
\small Zaczniemy od rozważenia funkcji $f(n, k)$ zwracającej ilość rozstawień $k$ nieszachujących się wież na planszy $A_{n}$, zdefiniowanej następująco:
\[
	A_{n} = \{(i, j) : 1 \le i \le j \le n\}
\]
Spróbujemy znaleźć jakąś zależność rekurencyjną jej dotyczącą. Rozbijmy problem $f(n, k)$ na dwa przypadki
\begin{enumerate}
\item{Pierwszy wiersz naszej planszy jest pusty. Wtedy mamy do rozstawienia $k$ wież na planszy $A_{n-1}$, czyli $f(n-1, k)$ możliwości }
\item{W pierwszym wierszu planszy stoi jedna wieża (więcej wszak nie może). Rozstawmy najpierw pozostałe $k-1$ wiersz w pozostałych $n-1$ wierszach (czyli planszy $A_{n-1}$). Jest $f(n-1, k-1)$ takich kombinacji. Następnie dostawiamy wieżę w pierwszym wierszu. $k-1$ pól jest już szachowanych, dlatego zostaje $n - (k - 1) = n + 1 - k$ miejsc. Razem daje to $(n + 1 - k)f(n-1, k-1)$ rozstawień.  }
\end{enumerate}
W ten sposób uzyskaliśmy, że
\[
f(n, k) = (n + 1 - k)f(n-1, k-1) + f(n-1, k)
\]
\\
Weźmy teraz funkcję $g(n, k) = \big\{ {n + 1 \atop n + 1 - k} \big\}$, z definicji liczb stirlinga:
\[
g(n, k) = \bigg\{ {n + 1 \atop n + 1 - k} \bigg\} = (n + 1 - k)\bigg\{ {n \atop n + 1 - k} \bigg\} + \bigg\{ {n \atop n - k} \bigg\} =
\]
\[
= (n + 1 - k)g(n-1, k-1) + g(n-1, k)
\]

Ojej! Wygląda podobnie. Ale to jeszcze nic nie znaczy, trzeba jeszcze przyjrzeć się warunkom brzegowym.

\[
f(n, n) = 1
\]
\[
f(n, 1) = \frac{n(n+1)}{2}
\]
\[
g(n, n) = \bigg\{ {n + 1 \atop 1} \bigg\} = 1
\]
\[
g(n, 1) = \bigg\{ {n + 1 \atop n } \bigg\} = \bigg( {n + 1 \atop 2} \bigg) = \frac{n(n+1)}{2}
\]
A to wspaniała wiadomość, bowiem mamy dwie funkcja o takiej samej zależności rekurencyjnej, posiadające identyczne warunki brzegowe. Zatem 
\[
f(n,k) = g(n,k) = \bigg\{ {n + 1 \atop n + 1 - k} \bigg\}
\]

Mając takie narzędzie właściwy problem rozwiąże się dość szybko. Ile jest możliwych rozstawień na danej planszy $B = \{ (i,j) : 1 \le i \le j \le n \}\setminus\{(1, n)\}$?
Weźmy wszystkie rozstawienia dla planszy $A_{n}a$ a następnie odejmijmy wszystkie te, w których na polu $(1, n)$ stoi jakaś wieża. Jest ich tyle co rozstawień $k-1$ wież na planszy $A_{n-2}$ (wieża stojąca w $(1,n)$ zabiera skrajny wiersz i kolumnę). Jest tego dokładnie rzecz biorąc
\[
  f(n, k) - f(n-2, k-1) = \bigg\{ {n + 1 \atop n + 1 - k} \bigg\} - \bigg\{ {n - 1 \atop n - k} \bigg\}
\]
co chcieliśmy udowodnić
\end{document}
