% Preambuła
\documentclass[a4paper,11pt]{article}
\usepackage[polish]{babel}
\usepackage[OT4]{fontenc}
\usepackage[utf8]{inputenc}
\usepackage{fancyhdr}
\usepackage{multirow}

\newtheorem{lemma}{Twierdzenie} % definicja środowiska Lemat


\pagestyle{fancy}
\lhead{MD'2010 - zadanie domowe nr 7}
\rhead{Wojciech Żółtak}

% Część główna
\begin{document}

Idąc po najmniejszej linii menalnego oporu skorzystamy tu z dość mocnego Tw. Kirchoffa, które brzmi następująco:

\begin{lemma} Twierdzenie Kirchoffa (macierzowe o drzewach) \footnote{Dowód tego niebywałego twierdzenia można znaleźć pod adresem: http://www.math.fau.edu/locke/Graphmat.htm \\ Wykorzystuje on Wzór Bineta-Cauchy'ego, którego dowód znajduje się pod adresem: http://www.lacim.uqam.ca/~lauve/courses/su2005-550/BS3.pdf} \\
Niech $G$ będzie grafem spójnym o $n$ wierzchołkach $v_1, v_2, ... v_n$. Niech A będzie laplasjanem grafu $G$. Wtedy liczba wszystkich drzew rozpinających grafu $G$ będzie równa dopełnieniu algebraicznemu dowolnego wyrazu macierzy $A$.
\end{lemma}

Laplasjan etykietowanego grafu, to macierz $n$ na $n$ zdefiniowana następująco:

\begin{displaymath}
a_{ij} = \left\{ \begin{array}{ll}
deg(v_i) & \textrm{dla $i=j$}\\
-1 & \textrm{gdy $i\not=j$ oraz $v_i$, $v_j$ połączone}\\
0 & \textrm{wpp}
\end{array} \right.
\end{displaymath}

Jest $m$ wierzchołków stopnia $n$ oraz $n$ wierzchołków stopnia $m$. Naturalnym etykietowaniem do zapisu naszego laplasjanu wydaje się (pogrubione liczby ,,wypełniają'' wolną przestrzeń bloków swoimi kopiami):

\begin{displaymath}
\left(
\begin{array}{ccc|ccc}
m          &             & \mathbf{0} &            &             &            \\
           & \ddots      &            &            & \mathbf{-1} &            \\
\mathbf{0} &             & m          &            &             &            \\ \hline
           &             &            & n          &             & \mathbf{0} \\
           & \mathbf{-1} &            &            & \ddots      &            \\
           &             &            & \mathbf{0} &             & n          

\end{array} \right)
\end{displaymath}

Ale zrobimy trochę inaczej, przyjmiemy sobie etykietowanie reprezentujące macierz:

\begin{displaymath}
\left(
\begin{array}{cc|ccc|ccc}
m      & -1     & 0          & \cdots      & 0          & -1         & \cdots      & -1         \\
-1     & n      & -1         & \cdots      & -1         & 0          & \cdots      & 0          \\ \hline
0      & -1     & m          &             & \mathbf{0} &            &             &            \\
\vdots & \vdots &            & \ddots      &            &            & \mathbf{-1} &            \\
0      & -1     & \mathbf{0} &             & m          &            &             &            \\ \hline 
-1     & 0      &            &             &            & n          &             & \mathbf{0} \\
\vdots & \vdots &            & \mathbf{-1} &            &            & \ddots      &            \\
-1     & 0      &            &             &            & \mathbf{0} &             & n          

\end{array} \right)
\end{displaymath}


Warto zauważyć, że zakładamy przy tym niezerowość $m$ oraz $n$ (wtedy problem się trywializuje) oraz, że w skrajnym przypadku część bloków znika (jest w nich $n-1$, lub $m-1$ elementów, czyli czasami zero). Nie przeszkadza nam to jednak w niczym, najwyżej zrobi się prościej obliczniowo.

Do policzenia dopełnienia algebraicznego musimy wybrać jakiś wiersz i kolumne i je usunąć. Robimy tak z wierszem nr $1$ oraz kolumną nr $2$ otrzymując:

\begin{displaymath}
\left(
\begin{array}{c|ccc|ccc}
-1     & -1         & \cdots      & -1         & 0          & \cdots      & 0          \\ \hline
0      & m          &             & \mathbf{0} &            &             &            \\
\vdots &            & \ddots      &            &            & \mathbf{-1} &            \\
0      & \mathbf{0} &             & m          &            &             &            \\ \hline 
-1     &            &             &            & n          &             & \mathbf{0} \\
\vdots &            & \mathbf{-1} &            &            & \ddots      &            \\
-1     &            &             &            & \mathbf{0} &             & n          

\end{array} \right)
\end{displaymath}

Pragniemy (mocno...) uzyskać macierz trójkątną, w czym przeszkadza las jedynek na lewo od bloku z $n$-ami. Szczęśliwie - możemy je wszystkie zredukować korzystając z pierwszego wiersza nic przy tym nie psując. Robimy tak i dochodzimy do:

\begin{displaymath}
\left(
\begin{array}{c|ccc|ccc}
-1     & -1         & \cdots      & -1         & 0          & \cdots      & 0          \\ \hline
0      & m          &             & \mathbf{0} &            &             &            \\
\vdots &            & \ddots      &            &            & \mathbf{-1} &            \\
0      & \mathbf{0} &             & m          &            &             &            \\ \hline 
0      &            &             &            & n          &             & \mathbf{0} \\
\vdots &            & \mathbf{0} &            &            & \ddots      &            \\
0      &            &             &            & \mathbf{0} &             & n          

\end{array} \right)
\end{displaymath}

Co jest macierzą trójkątną. Zatem wyznacznik jest iloczynem elementów na przekątnej równym $(-1)\cdot m^{n-1}n^{m-1}$. Dopełnienie każe nam to wymnożyć razy $(-1)^{a+b}$ gdzie $a$ i $b$ to numery usuniętego wiersza i kolumny. Zatem ostatecznie dopełnienie wynosi $(-1)^{1 + 1 + 2}\cdot m^{n-1}n^{m-1} = m^{n-1}n^{m-1}$.  Co chcieliśmy udowodnić.


\end{document}
