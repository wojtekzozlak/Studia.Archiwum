% Preambuła
\documentclass[a4paper,11pt]{article}
\usepackage[polish]{babel}
\usepackage[OT4]{fontenc}
\usepackage[utf8]{inputenc}
\usepackage{fancyhdr}

\pagestyle{fancy}
\lhead{PSiZI, opinia o przedmiocie}
\rhead{Wojciech Żółtak (wz292583)}

% Część główna
\begin{document}

Chciałbym napisać opinię nie tyle o samych wykładach, a o tym co dał mi ten przedmiot,
a co mnie we mnie bardzo zadziwiło. Nie widzę sensu w udowadnianiu, że
byłem na każdym spotkaniu (bo nie byłem) czy pisanie na siłę
 ,,co na spotkaniu było i dlaczego było to interesujące``.
Będzie od serca.
\\

Ale może od początku. Początek był dość smutny, bowiem PSiZI kojarzyło się
trochę ze szkoleniem z BHP. Będzie, będzie obowiązkowe i będzie nudne.
Ale stało się nieco inaczej. Bowiem przedmiot uświadomił mi, że\ldots uwaga!
- mam własne zdanie. Do tej pory myślałem o sobie bardziej jak o mocno
nieociosanym kamieniu, który nie jest w stanie skrystalizować stanowczej
opinii w jakimś temacie. A tu zaskoczenie - przychodzi na wykład osoba,
zaczyna coś mówić, a ja odczuwam głęboki, wewnętrzny sprzeciw! Co więcej,
jestem w stanie powiedzieć dlaczego i wyartykułować to koledze obok.

Bardzo dziwne uczucie\ldots

Posłużę się tu przykładem (a jednak będzie o wykładach!) Pana Andrzeja
Gąsiennicy-Samka (oby się to tak odmieniało\ldots), z którym nie zgadzam
się w kwestii, którą uczynił bardzo ważną częścią prezentacji, a mianowicie
,,brak doświadczenia nie może być usprawiedliwieniem``. Pan Andrzej
pomylił terminy ,,doświadczenia`` i czegoś w rodzaju ,,stażu pracy`` i z
zaangażowaniem godnym inkwizycji piętnował zasłanianie się brakiem doświadczenia
w momencie problemów wynikłych ze źle podjętych decyzji. Brzmiało to wszystko
jak jakaś młodzieńcza buta wymierzona w starszych, którzy ,,wiedzą lepiej``.
W każdym razie, Pan Andrzej wywołał u mnie raczej negatywne uczucia. I to chyba
jako jedyny z wszystkich prelegentów.
\\

A teraz nieco ogólniej, o tym które wykłady mi się szczególnie podobały i dlaczego.
Piszę z pamięci, a z nią bywa słabo, więc nie należy oczekiwać dokładnych szczegółów.
\\

Bardzo fajnie słuchało się o pracy informatyka za granicą i o tym, dlaczego
w Polsce tak nie będzie. Jakiś czas temu oglądałem odcinki polskiej edycji
Dragon's Den, które doskonale uwypuklały problem - Polacy panicznie boją się
ryzykować przy inwestowaniu.
\\

Bardzo ładnie w mojej świadomości odcisnął się wykład ze startupów. Rozpalał
wyobraźnię. Samo patrzenie na takiego totalnie zakręconego człowieka sprawiało
przyjemność. Już nawet nie chodzi o to, czy widzi się sens w jego działaniach,
czy nie - chodzi o przypomnienie, że praca to może być zabawa. A nawet w miarę
możliwości powinna.
\\

Bardzo bardzo pozytywne wrażenie zrobił na mnie wykład Prof. Blikle. Jeśli chodzi
o samą treść, to mniej - bo generalnie wydawała mi się dość oczywista, ale te
pączki! Przed wykładem dużo osób się śmiało, że prelegent powinien je z racji
swojej pozycji dostarczyć na wykład. Potem nam szczęki poopadały.
\\

Interesujący, choć nieco przerażający był też wykład Pana pracującego w
ArcaBit'cie. Generalnie, człowiek nie ma gdzie i jak dowiedzieć się co się dzieje
w takich firmach od kuchni. Z drugiej strony jak się słucha o powiązaniach
wirusów z biznesem, albo możliwościach botnetów,
to się trochę słabo na sercu robi. Jakoś tak człowiek chce
jeszcze myśleć, że żyje w pięknym, (prawie) idealnym świecie, a tu takie
burzenie światopoglądu\ldots
\\

Podsumowując, poza wspomnianą na samym początku prezentacją
Pana Andrzeja Gąsiennicy-Samka, żaden z wykładów nie wzbudził we mnie negatywnych
emocji. Może niektóre były mniej ciekawe (np. ten o Venture Capital), ale
wszystko jest kwestią osobistych preferencji. Nie chcę oceniać Pana Jana Madeja,
ani jego wykładów, 
bo jak napiszę, że okazał się bardzo fajną osobą, to będzie, że się podlizuje,
a szukać problemów na siłę nie zamierzam. Ogólnie rzecz biorąc, cały przedmiot okazał
się miłym zaskoczeniem i w jakiś sposób odskocznią od zwykłego toku zajęć.
Nie jest może niezbędny w toku nauczania informatyki, ale w fajny sposób dopełnia
przedmioty teoretyczne i techniczne. Nie mam jakichś konkretnych wskazówek
jak można by go prowadzić lepiej.
\\

Albo mam! Z chęcią bym posłuchał wykładu ,,jak założyć własną firmę``. W sensie
np. procedur prawnych, inkubatorów, zalet, wad itd.
\\

To wszystko! Dziękuję za uwagę.

\end{document}

