% Preambuła
\documentclass[a4paper,11pt]{article}
\usepackage[polish]{babel}
\usepackage[OT4]{fontenc}
\usepackage[utf8]{inputenc}
\usepackage{geometry}
\usepackage{fancyhdr}
\usepackage{bnf}

\geometry{lmargin=2cm}
\pagestyle{fancy}
\lhead{SiWP - zadanie domowe nr 1}
\rhead{Wojciech Żółtak}

% Część główna
\begin{document}

\begin{grammar}
  [(colon){$::=$}]
  [(period){.}]
  [(semicolon){$|$}]
  [(quote){\begin{bf}}{\end{bf}}]
  [(nonterminal){$}{$}]
\begin{tabular}{ r c l }
  $Num \ni$ <n> & : &  "0" ; "1" ; ... \\
  $Var \ni$ <x> & : & <x_1> ; <x_2> ; ... \\
  $Lab \ni$ <l> & : & <l_1> ; <l_2> ; ... \\
  $Expr \ni$ <e> & : & <n> ; <x> ; <e_1 + e_2> ; <e> \texttt{mod2} ; <e> \texttt{div2} \\
  $Dec \ni$ <d> & : & \texttt{var} <x = 0> ; <d_1 ; d_2> ; \texttt{proc} <x_1(x_2)\{I\}> \\
  $Instr \ni$ <I> & : & <x := e> ; <I_1 ; I_2> ; \texttt{skip} ; <l : \texttt{call } x(e)> ; \\
	& & <\texttt{begin } d \texttt{ in } I \texttt{ end}> ;
	<\texttt{if } (e \lessthan\greaterthan  0) \texttt{ then } \{I_1\} \texttt{ else } \{I_2\}> ; \\
	& & <\texttt{while } (e \lessthan\greaterthan 0) \texttt{ do } I \texttt{ done}> ;
	<\texttt{return to } l>
\end{tabular}
\end{grammar}

Definicje relacji semantycznych: 
\begin{eqnarray*}
Proc & = & Var \times Instr \times Env \\
\rho \in Env & = & Var \to Loc \\
s \in Store & = & Loc \to (\mathbf{Z} \cup Proc) \\
\mathbf{l} \in Loc & & \\
r_i \in T_I & & \\
\\
Conf_E & = & (Env \times Store \times Expr) \cup \mathbf{Z} \\
T_E & = & \mathbf{Z} \\
Conf_D & = & (Env \times Store \times Dec) \cup (Env \times Store) \\
T_D & = & Env \times Store \\
Conf_I & = & (Env \times Store \times Instr) \cup Store \cup (Store \times Lab) \\
T_I & = & Store \cup (Store \times Lab) \\
\end{eqnarray*}

{\bf Założenia:} \\
\begin{enumerate}
  \item{Semantyka wyrażeń jest dana z góry.}
  \item{Istnieje funkcja $newloc : Store \to Loc$ zwracająca nieużywaną lokację w danym składzie.}
  \item{$n$ występujące w regułach semantyki będzie, w zależności od kontekstu, ze zbioru $\mathbf{Z}$ bądź $Num$.}
\end{enumerate}

\newpage

{\bf Semantyka definicji:} \\

% var x = 0
\begin{tabular}{l}
\\
\hline
  $(\rho,\ s,\ \texttt{var } x = 0)
      \ \longrightarrow\
   (\rho[x \to \mathbf{l}],\ s[\mathbf{l} \to 0])$\\
\end{tabular}
dla $\mathbf{l} = newloc(s)$ \\
\\

% d1 ; d2
\begin{tabular}{l}
  $(\rho,\ s,\ d_1)
      \ \longrightarrow\ 
   (\rho',\ s')
      \ \ \ \ 
   (\rho',\ s',\ d_2)\
      \longrightarrow \
   (\rho'',\ s'')$\\
\hline
  $(\rho,\ s,\ d_1 ; d_2)
      \ \longrightarrow\ 
   (\rho'',\ s'')$\\
\end{tabular} \\ \\


% proc x1(x2){I}
\begin{tabular}{l}
  \\
\hline
  $(\rho,\ s,\ \texttt{proc } x_1(x_2)\{I\})
      \ \longrightarrow\ 
   (\rho[x_1 \to \mathbf{l}],\ s[\mathbf{l} \to (x_2,\ I,\ \rho)])$
\end{tabular} 
dla $\mathbf{l} = newloc(s)$
\\ \\

\ \\ \\

{\bf Semantyka instrukcji:} \\

% x := e
\begin{tabular}{l}
  $(\rho,\ s,\ e)\
      \longrightarrow\ 
   n$\\
\hline
  $(\rho,\ s,\ x := e)
      \ \longrightarrow\ 
   s[\rho(x) \to n] $ \\
\end{tabular} 
o ile $x \in Dom(\rho)$ 
\\ \\ 

% I1 ; I2 (ok)
\begin{tabular}{l}
  $(\rho,\ s,\ i_1)
      \ \longrightarrow\ 
   s'
      \ \ \ \ 
   (\rho,\ s',\ i_2)
      \ \longrightarrow\ 
   r_i$\\
\hline
  $(\rho,\ s,\ i_1 ; i_2)
      \ \longrightarrow\ 
   r_i$\\
\end{tabular} \\ \\

% I1 ; I2 (pierwsze return)
\begin{tabular}{l}
  $(\rho,\ s,\ i_1)
      \ \longrightarrow\ 
   (s',\ l)$ \\
\hline
  $(\rho,\ s,\ i_1 ; i_2)
      \ \longrightarrow\ 
   (s',\ l)$\\
\end{tabular} \\ \\


% skip
\begin{tabular}{l}
  \\
\hline
  $(\rho,\ s,\ \texttt{skip})
      \ \longrightarrow\ 
   s$\\
\end{tabular} \\ \\

% l: call x(e) (ok)
\begin{tabular}{l}
  $(\rho,\ s,\ e)
      \ \longrightarrow\ 
   n
      \ \ \ \ 
   (\rho'[y \to \mathbf{l}, \ x \to \rho(x)],\ s[\mathbf{l} \to n],\ I)
      \ \longrightarrow\ 
   s'$\\
\hline
  $(\rho,\ s,\ l : \texttt{call } x(e))
    \ \longrightarrow\ 
   s'$ \\
\end{tabular}
\begin{tabular}{l}
 o ile $s(\rho(x)) = (y,\ I,\ \rho')$ \\
 \tiny{} \\
 dla $\mathbf{l} = newloc(s)$ \\
\end{tabular} \\ \\


% l: call x(e) (return; pasuje label)
\begin{tabular}{l}
  $(\rho,\ s,\ e)
      \ \longrightarrow\ 
   n
      \ \ \ \ 
   (\rho'[y \to \mathbf{l}, \ x \to \rho(x)],\ s[\mathbf{l} \to n],\ I)
      \ \longrightarrow\ 
   (s', l)$\\
\hline
  $(\rho,\ s,\ l : \texttt{call } x(e))
    \ \longrightarrow\ 
   s'$ \\
\end{tabular}
\begin{tabular}{l}
 o ile $s(\rho(x)) = (y,\ I,\ \rho')$ \\
 \tiny{} \\
 dla $\mathbf{l} = newloc(s)$ \\
\end{tabular} \\ \\


% l: call x(e) (return; nie pasuje label)
\begin{tabular}{l}
  $(\rho,\ s,\ e)
      \ \longrightarrow\ 
   n
      \ \ \ \ 
   (\rho'[y \to \mathbf{l}, \ x \to \rho(x)],\ s[\mathbf{l} \to n],\ I)
      \ \longrightarrow\ 
   (s', l')$\\
\hline
  $(\rho,\ s,\ l : \texttt{call } x(e))
    \ \longrightarrow\ 
   (s',\ l')$ \\
\end{tabular}
\begin{tabular}{l}
 o ile $s(\rho(x)) = (y,\ I,\ \rho')$ \\
 oraz $l \neq l'$ \\
 \tiny{} \\
 dla $\mathbf{l} = newloc(s)$ \\
\end{tabular} \\ \\

% begin d in I end
\begin{tabular}{l}
  $(\rho,\ s,\ d)
      \ \longrightarrow\ 
   (\rho', s')\ 
      \ \ \ \
   (\rho',\ s',\ I)
      \ \longrightarrow\ 
   r_i $\\
\hline
  $(\rho,\ s,\ \texttt{begin } d \texttt{ in } I \texttt{ end})
      \ \longrightarrow\ 
   r_i $\\
\end{tabular} \\ \\


% if (e <> 0) then {I1} else {I2}  dla e <> 0
\begin{tabular}{l}
  $(\rho,\ s,\ e)
      \ \longrightarrow\ 
   n
      \ \ \ \ 
   (\rho,\ s,\ I_1)
      \ \longrightarrow\ 
   r_i$ \\
      \hline
  $(\rho,\ s,\ \texttt{if } (e <> 0) \texttt{ then } \{I_1\} \texttt{ else } \{I_2\})
      \ \longrightarrow\ 
   r_i $\\
\end{tabular}
o ile $n \neq 0$ \\
\\

% if (e <> 0) then {I1} else {I2}  dla e == 0
\begin{tabular}{l}
  $(\rho,\ s,\ e)
      \ \longrightarrow\ 
   n
      \ \ \ \ 
   (\rho,\ s,\ I_2)
      \ \longrightarrow\ 
   r_i$ \\
      \hline
  $(\rho,\ s,\ \texttt{if } (e <> 0) \texttt{ then } \{I_1\} \texttt{ else } \{I_2\})
      \ \longrightarrow\ 
   r_i $\\
\end{tabular}
o ile $n = 0$ \\
\\

% while (e <> 0) do I done (return)
\begin{tabular}{l}
  $(\rho,\ s,\ e)
      \ \longrightarrow\ 
   n
      \ \ \ \ 
   (\rho,\ s,\ I)
      \ \longrightarrow\
   (s',\ l)$ \\
\hline
  $(\rho,\ s,\ \texttt{while } (e <> 0) \texttt{ do } I \texttt{ done})
      \ \longrightarrow\ 
   (s',\ l)$ \\
\end{tabular}
o ile $n \neq  0$
\\ \\

% while (e <> 0) do I done (ok)
\begin{tabular}{l}
  $(\rho,\ s,\ e)
      \ \longrightarrow\ 
    n
      \ \ \ \ 
   (\rho,\ s,\ I)
      \ \longrightarrow\
    s'
      \ \ \ \ 
   (\rho,\ s',\ \texttt{while } (e <> 0) \texttt{ do } I \texttt{ done})
      \ \longrightarrow\ 
    r_i$ \\
\hline
  $(\rho,\ s,\ \texttt{while } (e <> 0) \texttt{ do } I \texttt{ done})
      \ \longrightarrow\ 
   r_i$ \\
\end{tabular}
o ile $n \neq  0$
 \\ \\

% while (e <> 0) do I done (fail)
\begin{tabular}{l}
  $(\rho,\ s,\ e)
      \ \longrightarrow\ 
    n$ \\
\hline
  $(\rho,\ s,\ \texttt{while } (e <> 0) \texttt{ do } I \texttt{ done})
      \ \longrightarrow\ 
   s$ \\
\end{tabular}
o ile $n = 0$
\\ \\


% return to l
\begin{tabular}{l}
  \\
\hline
  $(\rho,\ s,\ \texttt{return to } l)
      \ \longrightarrow\ 
   (s,\ l)$ \\
\end{tabular}

\end{document}

