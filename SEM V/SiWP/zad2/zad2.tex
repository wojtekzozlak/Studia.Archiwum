% Preambuła
\documentclass[a4paper,11pt]{article}
\usepackage[polish]{babel}
\usepackage[OT4]{fontenc}
\usepackage[utf8]{inputenc}
\usepackage{geometry}
\usepackage{fancyhdr}
\usepackage{bnf}
\usepackage{amsfonts}
\usepackage{stmaryrd}

\geometry{lmargin=2cm, tmargin=3cm, bmargin=3cm}
\pagestyle{fancy}
\lhead{SiWP - zadanie domowe nr 2}
\rhead{Wojciech Żółtak}

% Część główna
\begin{document}

\begin{grammar}
  [(colon){$::=$}]
  [(period){.}]
  [(semicolon){$|$}]
  [(quote){\begin{bf}}{\end{bf}}]
  [(nonterminal){$}{$}]
\begin{tabular}{ r c l }
  $Var \ni$ <x> & : & <x_1> ; <x_2> ; ... \\
  $Const \ni$ <n> & : &  ... ; \texttt{-1} ; \texttt{0} ; \texttt{1} ; ... \\
  $Exp \ni$ <e> & : & <n> ; <x> ; <e_1 + e_2> ; <e_1 - e_2> \\
  $Stm \ni$ <i> & : & <x := e> ; <i_1 ; i_2> ; \texttt{skip} ;
	<\texttt{if } e_1  = e_2 \texttt{ then } i_1 \texttt{ else } i_2> ; \\
	& & <\texttt{guard } e \texttt{ in } i \texttt{ end}> ;
	<\texttt{loop } i \texttt{ end}> \\
  $Prg \ni$ <p> & : & <i>
\end{tabular}
\end{grammar}

\begin{eqnarray*}
s \in State & = & Var \to \mathbf{Z_{\perp}} \\
g \in Guard & = & State \to \mathbf{Z_{\perp}} \\
N & : & Const \to \mathbf{Z_{\perp}}\\
\mathbb{E} & : & Exp \to State \to \mathbf{Z_{\perp}} \\
\mathbb{I} & : & Stm \to State \to Guard \to State \\
\mathbb{P} & : & Stm \to State
\end{eqnarray*} 

{\bf Założenia:}
\begin{enumerate}
  \item{Funkcja $N$ w sposób klasyczny konwertuje stałe na liczby.}
  \item{Operatory działań ($+$, $-$) są różne w zależności od kontekstu (definiują drzewo programu, lub operację na liczbach).}
\end{enumerate}
\ \\

{\bf Wyrażenia:}
\begin{eqnarray*}
  \mathbb{E} \llbracket n \rrbracket \ s & = & N(n) \\
  \mathbb{E} \llbracket x \rrbracket \ s & = & s(x)\\
  \mathbb{E} \llbracket e_1 + e_2 \rrbracket \ s & = & \mathbb{E} \llbracket e_1 \rrbracket s + \mathbb{E} \llbracket e_2 \rrbracket s \\
  \mathbb{E} \llbracket e_1 - e_2 \rrbracket \ s & = & \mathbb{E} \llbracket e_1 \rrbracket s - \mathbb{E} \llbracket e_2 \rrbracket s
\end{eqnarray*}

{\bf Instrukcje:}
\begin{eqnarray*}
  \mathbb{I} \llbracket x := e \rrbracket \ s \ g & = & s[x \to \mathbb{E} \llbracket e \rrbracket \ s] \\
  \mathbb{I} \llbracket i_1 ; i_2 \rrbracket \ s \ g & = & \mathbb{I} \llbracket i_2 \rrbracket \ 
    (\mathbb{I} \llbracket i_1 \rrbracket \ s \ g) \ g \\
  \mathbb{I} \llbracket \texttt{skip} \rrbracket \ s \ g & = & s \\
  \mathbb{I} \llbracket \texttt{if } e_1 = e_2 \texttt{ then } i_1 \texttt{ else } i_2 \rrbracket \ s \ g & = &
  \textsf{if } \mathbb{E} \llbracket e_1 \rrbracket \ s = \mathbb{E} \llbracket e_2 \rrbracket \ s 
\textsf{ then } \mathbb{I} \llbracket i_1 \rrbracket \ s \ g 
\textsf{ else } \mathbb{I} \llbracket i_2 \rrbracket \ s \ g \\
  \mathbb{I} \llbracket \texttt{guard } e \texttt{ in } i \texttt{ end} \rrbracket \ s \ g & = &
    \textsf{let } g' = (\lambda s. \textsf{if } \mathbb{E} \llbracket e \rrbracket \ s = 0
			\textsf{ then } 0 \textsf{ else } g(s) ) \\
     & &                \textsf{in } \mathbb{I} \llbracket i \rrbracket \ s \ g' \\
  \mathbb{I} \llbracket \texttt{loop } i \texttt{ end} \rrbracket & = & 
 Fix( \lambda f.\ \lambda s.\ \lambda g. \textsf{ if } g(s) = 0 \textsf{ then } s \textsf{ else } f (\mathbb{I} \llbracket i \rrbracket \ s \ g ) \ g )\\
    \\
\end{eqnarray*}

{\bf Programy:}
\begin{eqnarray*}
\mathbb{P} \llbracket i \rrbracket & = & \mathbb{I} \llbracket i \rrbracket (\lambda x. 0) (\lambda s. 1)
\end{eqnarray*}


\end{document}

